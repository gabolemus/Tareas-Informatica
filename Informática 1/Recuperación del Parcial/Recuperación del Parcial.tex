\documentclass[11pt,letterpaper]{article}

\usepackage[utf8]{inputenc}
\usepackage[spanish]{babel}
\usepackage{graphicx}
\usepackage{multirow}
\usepackage{amsmath}
\usepackage{amssymb}

\usepackage{etoolbox}
\AtBeginEnvironment{align}{\setcounter{equation}{0}}
\AtBeginEnvironment{eqnarray}{\setcounter{equation}{0}}

\usepackage{multicol}
\usepackage{amsthm}
\usepackage{cancel}
\usepackage[left=2cm,right=2cm,top=2cm,bottom=3cm]{geometry}

\linespread{1.1}

\begin{document}
\pagenumbering{gobble}

\begin{tabular}{l l}
\multirow{3}{*}{\includegraphics[width=2cm]{../../recursos/logo}} 
 & \Large Universidad del Istmo de Guatemala \\
 & \Large Facultad de Ingeniería \\
 & \Large Ing. en Sistemas \\
 & \Large Informática 1 \\
 & \Large Prof. Ernesto Rodríguez - erodriguez@unis.edu.gt \\
\end{tabular}
\\\\

\begin{center}
	\hrule
	\vspace{0.5cm}
	\huge{Corrección del Examen Parcial \#1} \\
	\vspace{0.1cm}
    \Large{Jeremy Cáceres y Gabriel Lemus}\\
    \vspace{0.05cm}
    \hrulefill
\end{center}
\vspace{0.05cm}

\emph{Instrucciones: Responda las preguntas que se le presentan a continuación. Asegúrese de brindar el procedimiento necesario para evaluar el razonamiento que le llevó a la respuesta. Puede hacer uso de cualquier material impreso para este examen. El examen debe resolverse de forma individual. Se le recuerda que la universidad no tolera el plagio y cualquier instancia de plagio será penalizado con la anulación de esta prueba, un reporte académico y la posible suspensión de ayuda financiera.}
\vspace{0.1cm}

\section*{Ejercicio \#1: Inducción (20\%)}
\noindent Demuestre las siguientes propiedades utilizando inducción. Puede hacer uso de la aritmética para dichas demostraciones. Asegúrese de indicar claramente el caso base, el caso inductivo, la hipótesis inductiva y cada paso del procedimiento.

\begin{enumerate}
\item $\forall n \geq 1. \ 2 * n$ es par
\\\\
\noindent \textbf{\large Caso Base:} \\
\noindent • $n=1$
\begin{proof}
\begin{eqnarray}
2 * 1 = 2 \\
2 \text{ es par}
\end{eqnarray}
\end{proof}

\vspace{0.1cm}

\noindent \textbf{\large Caso Inductivo:}\\
\noindent • $n=n+1$ \\
\noindent • \textbf{Hipótesis Inductiva: } $2 * n$ es par \\
\noindent • \emph{Demostrar que:} $2 * (n + 1)$ es par

\begin{proof}
\begin{align}
&2 * (n+1) && \text{Multiplicar por $2$.} \\
&(2*n) + 2 && \text{Sumarle 2 a un número no cambia si es par o no; se puede omitir.} \\
&2*n && \text{Hipótesis inductiva: 2 * n es par} \\
&2n \text{ es par}
\end{align}
\end{proof}


\item $\forall n \geq 4. \ 2^n < n!$, donde $n! = 1 * 2 * 3 * ... * (n-1) * n$
\\\\
\noindent \textbf{\large Caso Base:} \\
\noindent • $n=4$
\begin{proof}
\begin{eqnarray}
2^4 &<& 4! \\
16 &<& 24
\end{eqnarray}
\end{proof}

\vspace{0.1cm}

\noindent \textbf{\large Caso Inductivo:}\\
\noindent • $n=n+1$ \\
\noindent • \textbf{Hipótesis Inductiva: } $2^n < n!$ \\
\noindent • \emph{Demostrar que:} $2^{n+1} < (n+1)!$

\begin{proof}
\begin{align}
2^{n+1} &< (n+1)! && \text{Propiedad de los factoriales: $(n+1)! = n! * (n+1)$} \\
2^{n+1} &< n! * (n+1) && \text{Propiedad de la exponenciación: $2^{n+1} = 2^n * 2$} \\
2^n * 2 &< n! * (n+1) && \text{\small Usar un valor menor del lado izquierdo cumple con la desigualdad.} \\
2^n &< n! * (n+1) && \text{\small Usar un valor menor del lado derecho cumple con la desigualdad.} \\
2^n &< n! && \text{Hipótesis inductiva: $2^n < n!$.} \\
2^n &< n!
\end{align}
\end{proof}
\end{enumerate}
\pagebreak

\section*{Ejercicio \#2: Definiciones inductivas (60\%)}
\noindent Dar una definición inductiva para las siguientes funciones sobre los \emph{números naturales unarios}. Consejo, se le recomienda definir y utilizar la suma y multiplicación de los números naturales unarios:

\begin{enumerate}
\item La función factorial ($n!$) en donde $n! = 1 \otimes 2 \otimes 3 \otimes ... \otimes (n-1) \otimes n$
\[
	n! := \left\{
    \begin{array}{l l}
    	1 & \ \mbox{si } n=0 \lor n=1 \\
    	x! * \sigma(x) & \ \mbox{si } n=\sigma(x) \\
	\end{array}
    \right.
\]
\vspace{0.5cm}

\item La función resta ($\ominus$) en donde:
\begin{itemize}
\item[•] $a \ominus b = 0$ si $a \leq b$
\item[•] $a \ominus b = a - b$ de lo contrario
\end{itemize}
\[
	a \ominus b := \left\{
    \begin{array}{l l}
    	0 & \ \mbox{si } a \leq b \\
    	a & \ \mbox{si } b = 0 \\
    	1 & \ \mbox{si } a = \sigma(b) \\
    	(x \ominus b) \oplus 1 & \ \mbox{si } a=\sigma(x) \\
	\end{array}
    \right.
\]
\vspace{0.5cm}

\item La función sumatoria $\sum_{i}^{n}$ en donde $\sum_{i}^{n} = i \oplus (i \oplus 1) \oplus ... \oplus (n - 1) \oplus n$. En otras palabras, la suma de los números empezando por $i$ y terminando en $n$.
\[
	\sum_{i}^{n} := \left\{
    \begin{array}{l l}
    	0 & \ \mbox{si } n = 0 \\
    	1 & \ \mbox{si } n = 1 \\
    	\sum_{i}^{x}\ \oplus\ \sigma(x) & \ \mbox{si } n=\sigma(x) \wedge i \leq n\\
	\end{array}
    \right.
\]
\vspace{0.5cm}

\item La función exponente $a^b$ en donde $a^b = a \otimes a \otimes a ...$ ($b$ veces)
\[
	a^b := \left\{
    \begin{array}{l l}
    	1 & \ \mbox{si } b = 0 \\
    	a & \ \mbox{si } b = 1 \\
    	a \otimes a^x & \ \mbox{si } b = \sigma(x) \\
	\end{array}
    \right.
\]
\end{enumerate}
\pagebreak

\section*{Ejercicio \#3: Inducción para números unarios (20\%)}
\noindent A continuación se presenta la definición de la suma y multiplicación de números unarios:
\begin{multicols}{2}
\[
	a \oplus b := \left\{
    \begin{array}{l l}
    	a & \ \mbox{si } b=0 \\
    	b & \ \mbox{si } a=0 \\
        1 \oplus (x \oplus b) & \ \mbox{si } a=\sigma(x) \\
	\end{array}
    \right.
\]
\columnbreak
\\
\[
	a \otimes b := \left\{
    \begin{array}{l l}
    	0 & \ \mbox{si } a=0 \ \lor \ b=0 \\
    	a & \ \mbox{si } b=1 \\
        b & \ \mbox{si } a=1 \\
        a \oplus (a \otimes x) & \ \mbox{si } b=\sigma(x) \\
	\end{array}
    \right.
\]
\end{multicols}
\noindent Demostrar utilizando inducción que: $2 \otimes a = a \oplus a$. Puede utilizar una definición alterna (pero equivalente) de la suma o multiplicación si lo desea. Recuerde indicar claramente el caso base, el caso inductivo, la hipótesis inductiva y cada paso de la demostración.
\\\\
\noindent \textbf{\large Caso Base:} \\
\noindent • $a=0$
\begin{proof}
\begin{eqnarray}
2 \otimes 0 &=& 0 \oplus 0 \\
0 &=& 0
\end{eqnarray}
\end{proof}

\vspace{0.1cm}

\noindent \textbf{\large Caso Inductivo:}\\
\noindent • $a=\sigma(a)$ \\
\noindent • \textbf{Hipótesis Inductiva: } $2 \otimes a = a \oplus a$ \\
\noindent • \emph{Demostrar que:} $2 \otimes \sigma(a) = \sigma(a) \oplus \sigma(a)$

\begin{proof}
\begin{align}
2 \otimes \sigma(a) &= \sigma(a) \oplus \sigma(a) && \text{Definición de la multiplicación.} \\
2 \oplus (2 \otimes a) &= \sigma(a) \oplus \sigma(a) && \text{Hipótesis Inductiva: $2 \otimes a = a \oplus a$.} \\
2 \oplus (a \oplus a) &= \sigma(a) \oplus \sigma(a) && \text{Definición de la suma.} \\
2 \oplus (a \oplus a) &= \sigma(a \oplus \sigma(a)) && \text{La suma es conmutativa.} \\
2 \oplus (a \oplus a) &= \sigma(\sigma(a) \oplus a) && \text{Definición de la suma.} \\
2 \oplus (a \oplus a) &= \sigma(\sigma(a \oplus a)) && \text{$\sigma(\sigma(0)) = 2$} \\
2 \oplus (a \oplus a) &= 2 \oplus (a \oplus a)
\end{align}
\end{proof}

\end{document}