\documentclass[11pt,letterpaper]{article}

\usepackage[utf8]{inputenc}
\usepackage[spanish]{babel}
\usepackage{graphicx}
\usepackage{multirow}
\usepackage{amsmath}
\usepackage{amssymb}

\usepackage{etoolbox}
\AtBeginEnvironment{align}{\setcounter{equation}{0}}
\AtBeginEnvironment{eqnarray}{\setcounter{equation}{0}}

\usepackage{multicol}
\usepackage{amsthm}
\usepackage{cancel}
\usepackage[left=3cm,right=3cm,top=2cm,bottom=3cm]{geometry}

\linespread{1.1}

\begin{document}

\begin{tabular}{l l}
\multirow{3}{*}{\includegraphics[width=2cm]{../../recursos/logo}} 
 & \\
 & \LARGE Universidad del Istmo de Guatemala \\
 & \LARGE Facultad de Ingeniería \\
 & \LARGE Informática 1 \\
\end{tabular}
\\\\\\

\begin{center}
	\hrule
	\vspace{0.5cm}
	\huge{Laboratorio \#3} \\
	\vspace{0.1cm}
    \Large{Jeremy Cáceres y Gabriel Lemus}\\
    \vspace{0.05cm}
    \hrulefill
\end{center}
\vspace{0.05cm}

\section*{Ejercicio \#1}
\noindent Utilizando la definicion de suma ($\oplus$) para los números naturales unarios, llevar
a cabo la suma entre tres [$s(s(s(0)))$] y cuatro [$s(s(s(s(0))))$]. Debe elaborar todos
los pasos de forma explícita. Como referencia, se presenta nuevamente la definición de
suma para numeros natruales unarios:
\[
	n\oplus m := \left\{
    \begin{array}{l l}
    	m & \mbox{si } n=o \\
    	n & \mbox{si } m=o \\
    	\sigma(i\oplus m) & \mbox{si } n=\sigma(i) \\
    \end{array}
    \right.
\]

\noindent \emph{Demostración:}
\begin{align}
\sigma(\sigma(\sigma(0))) \oplus \sigma(\sigma(\sigma(\sigma(0))))& && \text{n\ =\ $\sigma(\sigma(\sigma(0)))$, \ i\ =\ $\sigma(\sigma(0))$} \\
\sigma[\sigma(\sigma(0)) \oplus \sigma(\sigma(\sigma(\sigma(0))))]& && \text{n\ =\ $\sigma(\sigma(0))$, \ i\ =\ $\sigma(0)$} \\
\sigma[\sigma[\sigma(0) \oplus \sigma(\sigma(\sigma(\sigma(0))))]]& && \text{n\ =\ $\sigma(0)$, \ i\ =\ $0$} \\
\sigma[\sigma[\sigma[0 \oplus \sigma(\sigma(\sigma(\sigma(0))))]]]& && \text{n\ =\ $0$ \ \ \ $\therefore 0 \oplus m=m$} \\
\sigma(\sigma(\sigma(\sigma(\sigma(\sigma(\sigma(0)))))))& && \text{$3+4=7$}
\end{align}

\section*{Ejercicio \#2}
\noindent Definir inductivamente una función para multiplicar ($\otimes$) números naturales unarios. \\
{\bf Consejo: }Puede apoyarse de la definición de suma estudiada durante la clase.
\\
\\
\emph{Definición:}
\[
	a\otimes b := \left\{
    \begin{array}{l l}
    	0 & \ \mbox{si } a=0 \ \lor \ b=0 \\
    	a & \ \mbox{si } b=\sigma(0) \\
        b & \ \mbox{si } a=\sigma(0) \\
        (a \otimes \sigma(i)) \oplus a & \ \mbox{si } b=\sigma(i) \\
	\end{array}
    \right.
\]

\section*{Ejercicio \#3}
\noindent Verifique que su definición de multiplicación es correcta multiplicando los siguientes valores:
\begin{enumerate}
	\item{$\sigma(\sigma(\sigma(0)))\otimes 0$}
	\begin{align}
	\sigma(\sigma(\sigma(0))) &\otimes 0 && \text{a\ =\ $\sigma(\sigma(\sigma(0)))$, b\ =\ $0$} \\
	&0 && \text{$\therefore \sigma(\sigma(\sigma(0))) \otimes 0 = 0$}
	\end{align}
	\\
	
	\item{$\sigma(\sigma(\sigma(0)))\otimes \sigma(0)$}
	\begin{align}
	\sigma(\sigma(\sigma(0))) &\otimes \sigma(0) && \text{a\ =\ $\sigma(\sigma(\sigma(0)))$, b\ =\ $\sigma(0)$} \\
	\sigma(\sigma(\sigma(&0))) && \text{$\therefore \ \sigma(\sigma(\sigma(0))) \otimes \sigma(0)=\sigma(\sigma(\sigma(0)))$}
	\end{align}		
	\\		
	
	\item{$\sigma(\sigma(\sigma(0)))\otimes \sigma(\sigma(0))$}
	\begin{align}
	&\sigma(\sigma(\sigma(0))) \otimes \sigma(\sigma(0)) && \text{a\ =\ $\sigma(\sigma(\sigma(0)))$, b\ =\ $\sigma(\sigma(0))$} \\
	&(\sigma(\sigma(\sigma(0))) \otimes \sigma(0)) \oplus \sigma(\sigma(\sigma(0))) && \text{a\ $\otimes\ \sigma(0)$\ =\ a} \\
	&\sigma(\sigma(\sigma(0))) \oplus \sigma(\sigma(\sigma(0))) && \text{Definición de la suma.} \\
	&\sigma(\sigma(\sigma(\sigma(\sigma(\sigma(0)))))) && \therefore 3\times2=6
	\end{align}		
	
\end{enumerate}
\vspace{0.05cm}

\section*{Ejercicio \#4}
\noindent Demostrar utilizando inducción:
\begin{enumerate}
\item{$a\oplus \sigma(\sigma(0))=\sigma(\sigma(a))$} \\
\\
\noindent \textbf{\large Caso Base:} \\
\noindent • $a=0$
\begin{proof}
\begin{eqnarray*}
0 \oplus \sigma(\sigma(0)) &=& \sigma(\sigma(0)) \\
\sigma(\sigma(0))&=& \sigma(\sigma(0))
\end{eqnarray*}
\end{proof}

\vspace{0.1cm}

\noindent \textbf{\large Caso Inductivo:}\\
\noindent • $a=\sigma(a)$ \\
\noindent • \textbf{Hipótesis Inductiva: } $a \oplus \sigma(\sigma(0)) = \sigma(\sigma(a))$ \\
\noindent • \emph{Demostrar que:} $\sigma(a) \oplus \sigma(\sigma(0)) = \sigma(\sigma(\sigma(a)))$

\begin{proof}
\begin{align}
\sigma(a) \oplus \sigma(\sigma(0)) &= \sigma(\sigma(\sigma(a))) && \text{Definición de la suma.}\\
\sigma[ a \oplus \sigma(\sigma(0))] &= \sigma(\sigma(\sigma(a))) && \text{La suma es conmutativa.}\\
\sigma[\sigma(\sigma(0)) \oplus a] &= \sigma(\sigma(\sigma(a))) && \text{Definición de la suma.}\\
\sigma[\sigma[\sigma(0) \oplus a]] &= \sigma(\sigma(\sigma(a))) && \text{Definición de la suma.}\\
\sigma[\sigma[\sigma[ 0 \oplus a]]] &= \sigma(\sigma(\sigma(a))) && \text{$0+a=a$}\\
\sigma(\sigma(\sigma(a))) &= \sigma(\sigma(\sigma(a)))
\end{align}
\end{proof}	

%%%%%%%%%%%%%%%%%%%%%%%%%%%%%%%%%%%%%%%%%%%%%%%%%%%%%%%%%%%%%%%%%%%%%%%%%%%%%%%%%%%%%%%%%%%%%%%%%%%%
%%%Problema2%%%

\item{$a \otimes b = b \otimes a$} \\
\\
\noindent \textbf{\large Caso Base:} \\
\noindent • $a=0$
\begin{proof}
\begin{eqnarray*}
0 \otimes b &=& b \otimes 0 \\
0 &=& 0
\end{eqnarray*}
\end{proof}

\vspace{0.1cm}

\noindent \textbf{\large Caso Inductivo:}\\
\noindent • $a=\sigma(a)$ \\
\noindent • \textbf{Hipótesis Inductiva: } $a \otimes b = b \otimes a$ \\
\noindent • \emph{Demostrar que:} $\sigma(a) \otimes b = b \otimes \sigma(a)$

\begin{proof}
\begin{align}
\sigma(a) \otimes b &= b \otimes \sigma(a) && \text{Definición de la multiplicación.}\\
\sigma(a) \otimes b &= (b \otimes a) \oplus b && \text{Por hipotesis inductiva.} \\
\sigma(a) \otimes b &= (a \otimes b) \oplus b && \text{Definición de la multiplicación.}\\
(a \otimes b) \oplus b &= (a \otimes b) \oplus b
\end{align}
\end{proof}
\pagebreak
%%%%%%%%%%%%%%%%%%%%%%%%%%%%%%%%%%%%%%%%%%%%%%%%%%%%%%%%%%%%%%%%%%%%%%%%%%%%%%%%%%%%%%%%%%%%%%%%%%%%
%%%Problema3%%%

\item{$a \otimes (b \otimes c)=(a\otimes b)\otimes c$} \\
\\
\noindent \textbf{\large Caso Base:}\\
\noindent • $c=0$
\begin{proof}
\begin{align*}
a \otimes (b \otimes 0) &= (a \otimes b) \otimes 0 && \text{i\ =\ a $\otimes$ b}\\
a \otimes 0 &= i \otimes 0 \\
0 &= 0
\end{align*}
\end{proof}

\vspace{0.1cm}

\noindent \textbf{\large Caso Inductivo:}\\
\noindent • $c=\sigma(c)$ \\
\noindent • \textbf{Hipótesis Inductiva: } $a \otimes (b \otimes c)=(a\otimes b)\otimes c$ \\
\noindent • \emph{Demostrar que:} $a \otimes (b \otimes \sigma(c))=(a \otimes b)\otimes \sigma(c)$

\begin{proof}
\begin{align}
a \otimes (b \otimes \sigma(c))&=(a \otimes b)\otimes \sigma(c) && \text{Definición de la multiplicación.} \\
a \otimes (b \otimes \sigma(c))&=((a \otimes b)\otimes c) \oplus (a \otimes b) && \text{Por hipotesis inductiva.} \\
a \otimes (b \otimes \sigma(c))&= (a \otimes (b \otimes c)) \oplus (a \otimes b) && \text{La suma es conmutativa.} \\
a \otimes (b \otimes \sigma(c))&= (a \otimes b) \oplus (a \otimes (b \otimes c)) && \text{La multiplicación es distributiva.} \\
a \otimes (b \otimes \sigma(c))&= a \otimes (b \oplus (b \otimes c)) && \text{La suma es conmutativa.} \\
a \otimes (b \otimes \sigma(c))&= a \otimes ((b \otimes c) \oplus b) && \text{Definición de la multiplicación.}\\
a \otimes (b \otimes \sigma(c))&= a \otimes (b \otimes \sigma(c))
\end{align}
\end{proof}

%%%%%%%%%%%%%%%%%%%%%%%%%%%%%%%%%%%%%%%%%%%%%%%%%%%%%%%%%%%%%%%%%%%%%%%%%%%%%%%%%%%%%%%%%%%%%%%%%%%%
%%%Problema4%%%

\item{$(a\oplus b)\otimes c = (a\otimes c) \oplus (b \otimes c)$} \\
\\
\noindent \textbf{\large Caso Base:}\\
\noindent • $c=0$
\begin{proof}
\begin{align*}
(a \oplus b) \otimes 0 &= (a \otimes 0) \oplus (b \otimes 0) && \text{i\ =\ a\ $\oplus$\ b} \\
i \otimes 0 &= 0 \oplus 0 \\
0 &= 0
\end{align*}
\end{proof}

\vspace{0.1cm}

\noindent \textbf{\large Caso Inductivo:}\\
\noindent • $c=\sigma(c)$ \\
\noindent • \textbf{Hipótesis Inductiva: } $(a\oplus b)\otimes c = (a\otimes c) \oplus (b \otimes c)$ \\
\noindent • \emph{Demostrar que:} $(a\oplus b)\otimes \sigma(c) = (a\otimes \sigma(c)) \oplus (b \otimes \sigma(c))$

\begin{proof}
\begin{align*}
&1)\ (a\oplus b)\otimes \sigma(c) = (a\otimes \sigma(c)) \oplus (b \otimes \sigma(c)) && \text{Definición de la multiplicación.} \\
&2)\ (a \oplus b) \otimes c \oplus (a \oplus b) = (a\otimes \sigma(c)) \oplus (b \otimes \sigma(c)) && \text{Por hipótesis de inducción.} \\
&3)\ (a \otimes b) \oplus (b \otimes c) \oplus (a \oplus b) = (a\otimes \sigma(c)) \oplus (b \otimes \sigma(c)) && \text{La suma es asociativa y conmutativa.} \\
&4)\ ((a \otimes c) \oplus a) \oplus ((b \otimes c) \oplus b) = (a\otimes \sigma(c)) \oplus (b \otimes \sigma(c)) && \text{Definición de la multiplicación.} \\
&5)\ (a \otimes \sigma(c)) \oplus (b \otimes \sigma(c)) = (a\otimes \sigma(c)) \oplus (b \otimes \sigma(c))
\end{align*}
\end{proof}

\end{enumerate}

\end{document}